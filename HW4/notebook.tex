
% Default to the notebook output style

    


% Inherit from the specified cell style.




    
\documentclass[11pt]{article}

    
    
    \usepackage[T1]{fontenc}
    % Nicer default font (+ math font) than Computer Modern for most use cases
    \usepackage{mathpazo}

    % Basic figure setup, for now with no caption control since it's done
    % automatically by Pandoc (which extracts ![](path) syntax from Markdown).
    \usepackage{graphicx}
    % We will generate all images so they have a width \maxwidth. This means
    % that they will get their normal width if they fit onto the page, but
    % are scaled down if they would overflow the margins.
    \makeatletter
    \def\maxwidth{\ifdim\Gin@nat@width>\linewidth\linewidth
    \else\Gin@nat@width\fi}
    \makeatother
    \let\Oldincludegraphics\includegraphics
    % Set max figure width to be 80% of text width, for now hardcoded.
    \renewcommand{\includegraphics}[1]{\Oldincludegraphics[width=.8\maxwidth]{#1}}
    % Ensure that by default, figures have no caption (until we provide a
    % proper Figure object with a Caption API and a way to capture that
    % in the conversion process - todo).
    \usepackage{caption}
    \DeclareCaptionLabelFormat{nolabel}{}
    \captionsetup{labelformat=nolabel}

    \usepackage{adjustbox} % Used to constrain images to a maximum size 
    \usepackage{xcolor} % Allow colors to be defined
    \usepackage{enumerate} % Needed for markdown enumerations to work
    \usepackage{geometry} % Used to adjust the document margins
    \usepackage{amsmath} % Equations
    \usepackage{amssymb} % Equations
    \usepackage{textcomp} % defines textquotesingle
    % Hack from http://tex.stackexchange.com/a/47451/13684:
    \AtBeginDocument{%
        \def\PYZsq{\textquotesingle}% Upright quotes in Pygmentized code
    }
    \usepackage{upquote} % Upright quotes for verbatim code
    \usepackage{eurosym} % defines \euro
    \usepackage[mathletters]{ucs} % Extended unicode (utf-8) support
    \usepackage[utf8x]{inputenc} % Allow utf-8 characters in the tex document
    \usepackage{fancyvrb} % verbatim replacement that allows latex
    \usepackage{grffile} % extends the file name processing of package graphics 
                         % to support a larger range 
    % The hyperref package gives us a pdf with properly built
    % internal navigation ('pdf bookmarks' for the table of contents,
    % internal cross-reference links, web links for URLs, etc.)
    \usepackage{hyperref}
    \usepackage{longtable} % longtable support required by pandoc >1.10
    \usepackage{booktabs}  % table support for pandoc > 1.12.2
    \usepackage[inline]{enumitem} % IRkernel/repr support (it uses the enumerate* environment)
    \usepackage[normalem]{ulem} % ulem is needed to support strikethroughs (\sout)
                                % normalem makes italics be italics, not underlines
    

    
    
    % Colors for the hyperref package
    \definecolor{urlcolor}{rgb}{0,.145,.698}
    \definecolor{linkcolor}{rgb}{.71,0.21,0.01}
    \definecolor{citecolor}{rgb}{.12,.54,.11}

    % ANSI colors
    \definecolor{ansi-black}{HTML}{3E424D}
    \definecolor{ansi-black-intense}{HTML}{282C36}
    \definecolor{ansi-red}{HTML}{E75C58}
    \definecolor{ansi-red-intense}{HTML}{B22B31}
    \definecolor{ansi-green}{HTML}{00A250}
    \definecolor{ansi-green-intense}{HTML}{007427}
    \definecolor{ansi-yellow}{HTML}{DDB62B}
    \definecolor{ansi-yellow-intense}{HTML}{B27D12}
    \definecolor{ansi-blue}{HTML}{208FFB}
    \definecolor{ansi-blue-intense}{HTML}{0065CA}
    \definecolor{ansi-magenta}{HTML}{D160C4}
    \definecolor{ansi-magenta-intense}{HTML}{A03196}
    \definecolor{ansi-cyan}{HTML}{60C6C8}
    \definecolor{ansi-cyan-intense}{HTML}{258F8F}
    \definecolor{ansi-white}{HTML}{C5C1B4}
    \definecolor{ansi-white-intense}{HTML}{A1A6B2}

    % commands and environments needed by pandoc snippets
    % extracted from the output of `pandoc -s`
    \providecommand{\tightlist}{%
      \setlength{\itemsep}{0pt}\setlength{\parskip}{0pt}}
    \DefineVerbatimEnvironment{Highlighting}{Verbatim}{commandchars=\\\{\}}
    % Add ',fontsize=\small' for more characters per line
    \newenvironment{Shaded}{}{}
    \newcommand{\KeywordTok}[1]{\textcolor[rgb]{0.00,0.44,0.13}{\textbf{{#1}}}}
    \newcommand{\DataTypeTok}[1]{\textcolor[rgb]{0.56,0.13,0.00}{{#1}}}
    \newcommand{\DecValTok}[1]{\textcolor[rgb]{0.25,0.63,0.44}{{#1}}}
    \newcommand{\BaseNTok}[1]{\textcolor[rgb]{0.25,0.63,0.44}{{#1}}}
    \newcommand{\FloatTok}[1]{\textcolor[rgb]{0.25,0.63,0.44}{{#1}}}
    \newcommand{\CharTok}[1]{\textcolor[rgb]{0.25,0.44,0.63}{{#1}}}
    \newcommand{\StringTok}[1]{\textcolor[rgb]{0.25,0.44,0.63}{{#1}}}
    \newcommand{\CommentTok}[1]{\textcolor[rgb]{0.38,0.63,0.69}{\textit{{#1}}}}
    \newcommand{\OtherTok}[1]{\textcolor[rgb]{0.00,0.44,0.13}{{#1}}}
    \newcommand{\AlertTok}[1]{\textcolor[rgb]{1.00,0.00,0.00}{\textbf{{#1}}}}
    \newcommand{\FunctionTok}[1]{\textcolor[rgb]{0.02,0.16,0.49}{{#1}}}
    \newcommand{\RegionMarkerTok}[1]{{#1}}
    \newcommand{\ErrorTok}[1]{\textcolor[rgb]{1.00,0.00,0.00}{\textbf{{#1}}}}
    \newcommand{\NormalTok}[1]{{#1}}
    
    % Additional commands for more recent versions of Pandoc
    \newcommand{\ConstantTok}[1]{\textcolor[rgb]{0.53,0.00,0.00}{{#1}}}
    \newcommand{\SpecialCharTok}[1]{\textcolor[rgb]{0.25,0.44,0.63}{{#1}}}
    \newcommand{\VerbatimStringTok}[1]{\textcolor[rgb]{0.25,0.44,0.63}{{#1}}}
    \newcommand{\SpecialStringTok}[1]{\textcolor[rgb]{0.73,0.40,0.53}{{#1}}}
    \newcommand{\ImportTok}[1]{{#1}}
    \newcommand{\DocumentationTok}[1]{\textcolor[rgb]{0.73,0.13,0.13}{\textit{{#1}}}}
    \newcommand{\AnnotationTok}[1]{\textcolor[rgb]{0.38,0.63,0.69}{\textbf{\textit{{#1}}}}}
    \newcommand{\CommentVarTok}[1]{\textcolor[rgb]{0.38,0.63,0.69}{\textbf{\textit{{#1}}}}}
    \newcommand{\VariableTok}[1]{\textcolor[rgb]{0.10,0.09,0.49}{{#1}}}
    \newcommand{\ControlFlowTok}[1]{\textcolor[rgb]{0.00,0.44,0.13}{\textbf{{#1}}}}
    \newcommand{\OperatorTok}[1]{\textcolor[rgb]{0.40,0.40,0.40}{{#1}}}
    \newcommand{\BuiltInTok}[1]{{#1}}
    \newcommand{\ExtensionTok}[1]{{#1}}
    \newcommand{\PreprocessorTok}[1]{\textcolor[rgb]{0.74,0.48,0.00}{{#1}}}
    \newcommand{\AttributeTok}[1]{\textcolor[rgb]{0.49,0.56,0.16}{{#1}}}
    \newcommand{\InformationTok}[1]{\textcolor[rgb]{0.38,0.63,0.69}{\textbf{\textit{{#1}}}}}
    \newcommand{\WarningTok}[1]{\textcolor[rgb]{0.38,0.63,0.69}{\textbf{\textit{{#1}}}}}
    
    
    % Define a nice break command that doesn't care if a line doesn't already
    % exist.
    \def\br{\hspace*{\fill} \\* }
    % Math Jax compatability definitions
    \def\gt{>}
    \def\lt{<}
    % Document parameters
    \title{HW4\_B87772}
    
    
    

    % Pygments definitions
    
\makeatletter
\def\PY@reset{\let\PY@it=\relax \let\PY@bf=\relax%
    \let\PY@ul=\relax \let\PY@tc=\relax%
    \let\PY@bc=\relax \let\PY@ff=\relax}
\def\PY@tok#1{\csname PY@tok@#1\endcsname}
\def\PY@toks#1+{\ifx\relax#1\empty\else%
    \PY@tok{#1}\expandafter\PY@toks\fi}
\def\PY@do#1{\PY@bc{\PY@tc{\PY@ul{%
    \PY@it{\PY@bf{\PY@ff{#1}}}}}}}
\def\PY#1#2{\PY@reset\PY@toks#1+\relax+\PY@do{#2}}

\expandafter\def\csname PY@tok@w\endcsname{\def\PY@tc##1{\textcolor[rgb]{0.73,0.73,0.73}{##1}}}
\expandafter\def\csname PY@tok@c\endcsname{\let\PY@it=\textit\def\PY@tc##1{\textcolor[rgb]{0.25,0.50,0.50}{##1}}}
\expandafter\def\csname PY@tok@cp\endcsname{\def\PY@tc##1{\textcolor[rgb]{0.74,0.48,0.00}{##1}}}
\expandafter\def\csname PY@tok@k\endcsname{\let\PY@bf=\textbf\def\PY@tc##1{\textcolor[rgb]{0.00,0.50,0.00}{##1}}}
\expandafter\def\csname PY@tok@kp\endcsname{\def\PY@tc##1{\textcolor[rgb]{0.00,0.50,0.00}{##1}}}
\expandafter\def\csname PY@tok@kt\endcsname{\def\PY@tc##1{\textcolor[rgb]{0.69,0.00,0.25}{##1}}}
\expandafter\def\csname PY@tok@o\endcsname{\def\PY@tc##1{\textcolor[rgb]{0.40,0.40,0.40}{##1}}}
\expandafter\def\csname PY@tok@ow\endcsname{\let\PY@bf=\textbf\def\PY@tc##1{\textcolor[rgb]{0.67,0.13,1.00}{##1}}}
\expandafter\def\csname PY@tok@nb\endcsname{\def\PY@tc##1{\textcolor[rgb]{0.00,0.50,0.00}{##1}}}
\expandafter\def\csname PY@tok@nf\endcsname{\def\PY@tc##1{\textcolor[rgb]{0.00,0.00,1.00}{##1}}}
\expandafter\def\csname PY@tok@nc\endcsname{\let\PY@bf=\textbf\def\PY@tc##1{\textcolor[rgb]{0.00,0.00,1.00}{##1}}}
\expandafter\def\csname PY@tok@nn\endcsname{\let\PY@bf=\textbf\def\PY@tc##1{\textcolor[rgb]{0.00,0.00,1.00}{##1}}}
\expandafter\def\csname PY@tok@ne\endcsname{\let\PY@bf=\textbf\def\PY@tc##1{\textcolor[rgb]{0.82,0.25,0.23}{##1}}}
\expandafter\def\csname PY@tok@nv\endcsname{\def\PY@tc##1{\textcolor[rgb]{0.10,0.09,0.49}{##1}}}
\expandafter\def\csname PY@tok@no\endcsname{\def\PY@tc##1{\textcolor[rgb]{0.53,0.00,0.00}{##1}}}
\expandafter\def\csname PY@tok@nl\endcsname{\def\PY@tc##1{\textcolor[rgb]{0.63,0.63,0.00}{##1}}}
\expandafter\def\csname PY@tok@ni\endcsname{\let\PY@bf=\textbf\def\PY@tc##1{\textcolor[rgb]{0.60,0.60,0.60}{##1}}}
\expandafter\def\csname PY@tok@na\endcsname{\def\PY@tc##1{\textcolor[rgb]{0.49,0.56,0.16}{##1}}}
\expandafter\def\csname PY@tok@nt\endcsname{\let\PY@bf=\textbf\def\PY@tc##1{\textcolor[rgb]{0.00,0.50,0.00}{##1}}}
\expandafter\def\csname PY@tok@nd\endcsname{\def\PY@tc##1{\textcolor[rgb]{0.67,0.13,1.00}{##1}}}
\expandafter\def\csname PY@tok@s\endcsname{\def\PY@tc##1{\textcolor[rgb]{0.73,0.13,0.13}{##1}}}
\expandafter\def\csname PY@tok@sd\endcsname{\let\PY@it=\textit\def\PY@tc##1{\textcolor[rgb]{0.73,0.13,0.13}{##1}}}
\expandafter\def\csname PY@tok@si\endcsname{\let\PY@bf=\textbf\def\PY@tc##1{\textcolor[rgb]{0.73,0.40,0.53}{##1}}}
\expandafter\def\csname PY@tok@se\endcsname{\let\PY@bf=\textbf\def\PY@tc##1{\textcolor[rgb]{0.73,0.40,0.13}{##1}}}
\expandafter\def\csname PY@tok@sr\endcsname{\def\PY@tc##1{\textcolor[rgb]{0.73,0.40,0.53}{##1}}}
\expandafter\def\csname PY@tok@ss\endcsname{\def\PY@tc##1{\textcolor[rgb]{0.10,0.09,0.49}{##1}}}
\expandafter\def\csname PY@tok@sx\endcsname{\def\PY@tc##1{\textcolor[rgb]{0.00,0.50,0.00}{##1}}}
\expandafter\def\csname PY@tok@m\endcsname{\def\PY@tc##1{\textcolor[rgb]{0.40,0.40,0.40}{##1}}}
\expandafter\def\csname PY@tok@gh\endcsname{\let\PY@bf=\textbf\def\PY@tc##1{\textcolor[rgb]{0.00,0.00,0.50}{##1}}}
\expandafter\def\csname PY@tok@gu\endcsname{\let\PY@bf=\textbf\def\PY@tc##1{\textcolor[rgb]{0.50,0.00,0.50}{##1}}}
\expandafter\def\csname PY@tok@gd\endcsname{\def\PY@tc##1{\textcolor[rgb]{0.63,0.00,0.00}{##1}}}
\expandafter\def\csname PY@tok@gi\endcsname{\def\PY@tc##1{\textcolor[rgb]{0.00,0.63,0.00}{##1}}}
\expandafter\def\csname PY@tok@gr\endcsname{\def\PY@tc##1{\textcolor[rgb]{1.00,0.00,0.00}{##1}}}
\expandafter\def\csname PY@tok@ge\endcsname{\let\PY@it=\textit}
\expandafter\def\csname PY@tok@gs\endcsname{\let\PY@bf=\textbf}
\expandafter\def\csname PY@tok@gp\endcsname{\let\PY@bf=\textbf\def\PY@tc##1{\textcolor[rgb]{0.00,0.00,0.50}{##1}}}
\expandafter\def\csname PY@tok@go\endcsname{\def\PY@tc##1{\textcolor[rgb]{0.53,0.53,0.53}{##1}}}
\expandafter\def\csname PY@tok@gt\endcsname{\def\PY@tc##1{\textcolor[rgb]{0.00,0.27,0.87}{##1}}}
\expandafter\def\csname PY@tok@err\endcsname{\def\PY@bc##1{\setlength{\fboxsep}{0pt}\fcolorbox[rgb]{1.00,0.00,0.00}{1,1,1}{\strut ##1}}}
\expandafter\def\csname PY@tok@kc\endcsname{\let\PY@bf=\textbf\def\PY@tc##1{\textcolor[rgb]{0.00,0.50,0.00}{##1}}}
\expandafter\def\csname PY@tok@kd\endcsname{\let\PY@bf=\textbf\def\PY@tc##1{\textcolor[rgb]{0.00,0.50,0.00}{##1}}}
\expandafter\def\csname PY@tok@kn\endcsname{\let\PY@bf=\textbf\def\PY@tc##1{\textcolor[rgb]{0.00,0.50,0.00}{##1}}}
\expandafter\def\csname PY@tok@kr\endcsname{\let\PY@bf=\textbf\def\PY@tc##1{\textcolor[rgb]{0.00,0.50,0.00}{##1}}}
\expandafter\def\csname PY@tok@bp\endcsname{\def\PY@tc##1{\textcolor[rgb]{0.00,0.50,0.00}{##1}}}
\expandafter\def\csname PY@tok@fm\endcsname{\def\PY@tc##1{\textcolor[rgb]{0.00,0.00,1.00}{##1}}}
\expandafter\def\csname PY@tok@vc\endcsname{\def\PY@tc##1{\textcolor[rgb]{0.10,0.09,0.49}{##1}}}
\expandafter\def\csname PY@tok@vg\endcsname{\def\PY@tc##1{\textcolor[rgb]{0.10,0.09,0.49}{##1}}}
\expandafter\def\csname PY@tok@vi\endcsname{\def\PY@tc##1{\textcolor[rgb]{0.10,0.09,0.49}{##1}}}
\expandafter\def\csname PY@tok@vm\endcsname{\def\PY@tc##1{\textcolor[rgb]{0.10,0.09,0.49}{##1}}}
\expandafter\def\csname PY@tok@sa\endcsname{\def\PY@tc##1{\textcolor[rgb]{0.73,0.13,0.13}{##1}}}
\expandafter\def\csname PY@tok@sb\endcsname{\def\PY@tc##1{\textcolor[rgb]{0.73,0.13,0.13}{##1}}}
\expandafter\def\csname PY@tok@sc\endcsname{\def\PY@tc##1{\textcolor[rgb]{0.73,0.13,0.13}{##1}}}
\expandafter\def\csname PY@tok@dl\endcsname{\def\PY@tc##1{\textcolor[rgb]{0.73,0.13,0.13}{##1}}}
\expandafter\def\csname PY@tok@s2\endcsname{\def\PY@tc##1{\textcolor[rgb]{0.73,0.13,0.13}{##1}}}
\expandafter\def\csname PY@tok@sh\endcsname{\def\PY@tc##1{\textcolor[rgb]{0.73,0.13,0.13}{##1}}}
\expandafter\def\csname PY@tok@s1\endcsname{\def\PY@tc##1{\textcolor[rgb]{0.73,0.13,0.13}{##1}}}
\expandafter\def\csname PY@tok@mb\endcsname{\def\PY@tc##1{\textcolor[rgb]{0.40,0.40,0.40}{##1}}}
\expandafter\def\csname PY@tok@mf\endcsname{\def\PY@tc##1{\textcolor[rgb]{0.40,0.40,0.40}{##1}}}
\expandafter\def\csname PY@tok@mh\endcsname{\def\PY@tc##1{\textcolor[rgb]{0.40,0.40,0.40}{##1}}}
\expandafter\def\csname PY@tok@mi\endcsname{\def\PY@tc##1{\textcolor[rgb]{0.40,0.40,0.40}{##1}}}
\expandafter\def\csname PY@tok@il\endcsname{\def\PY@tc##1{\textcolor[rgb]{0.40,0.40,0.40}{##1}}}
\expandafter\def\csname PY@tok@mo\endcsname{\def\PY@tc##1{\textcolor[rgb]{0.40,0.40,0.40}{##1}}}
\expandafter\def\csname PY@tok@ch\endcsname{\let\PY@it=\textit\def\PY@tc##1{\textcolor[rgb]{0.25,0.50,0.50}{##1}}}
\expandafter\def\csname PY@tok@cm\endcsname{\let\PY@it=\textit\def\PY@tc##1{\textcolor[rgb]{0.25,0.50,0.50}{##1}}}
\expandafter\def\csname PY@tok@cpf\endcsname{\let\PY@it=\textit\def\PY@tc##1{\textcolor[rgb]{0.25,0.50,0.50}{##1}}}
\expandafter\def\csname PY@tok@c1\endcsname{\let\PY@it=\textit\def\PY@tc##1{\textcolor[rgb]{0.25,0.50,0.50}{##1}}}
\expandafter\def\csname PY@tok@cs\endcsname{\let\PY@it=\textit\def\PY@tc##1{\textcolor[rgb]{0.25,0.50,0.50}{##1}}}

\def\PYZbs{\char`\\}
\def\PYZus{\char`\_}
\def\PYZob{\char`\{}
\def\PYZcb{\char`\}}
\def\PYZca{\char`\^}
\def\PYZam{\char`\&}
\def\PYZlt{\char`\<}
\def\PYZgt{\char`\>}
\def\PYZsh{\char`\#}
\def\PYZpc{\char`\%}
\def\PYZdl{\char`\$}
\def\PYZhy{\char`\-}
\def\PYZsq{\char`\'}
\def\PYZdq{\char`\"}
\def\PYZti{\char`\~}
% for compatibility with earlier versions
\def\PYZat{@}
\def\PYZlb{[}
\def\PYZrb{]}
\makeatother


    % Exact colors from NB
    \definecolor{incolor}{rgb}{0.0, 0.0, 0.5}
    \definecolor{outcolor}{rgb}{0.545, 0.0, 0.0}



    
    % Prevent overflowing lines due to hard-to-break entities
    \sloppy 
    % Setup hyperref package
    \hypersetup{
      breaklinks=true,  % so long urls are correctly broken across lines
      colorlinks=true,
      urlcolor=urlcolor,
      linkcolor=linkcolor,
      citecolor=citecolor,
      }
    % Slightly bigger margins than the latex defaults
    
    \geometry{verbose,tmargin=1in,bmargin=1in,lmargin=1in,rmargin=1in}
    
    

    \begin{document}
    
    
    \maketitle
    
    

    
    EX1: In this exercise I implemented Bucket sort algorithm. It is a
distribution sort that works by arranging elements into several
`buckets' which are then sorted using another sort(I used insertion
sort) and merged into a sorted list. To compare bucket sort algorithm I
chose the python's built-in algorithm Timsort.

    \begin{Verbatim}[commandchars=\\\{\}]
{\color{incolor}In [{\color{incolor}1}]:} \PY{k+kn}{import} \PY{n+nn}{random}\PY{o}{,} \PY{n+nn}{sys}\PY{o}{,} \PY{n+nn}{time}\PY{o}{,} \PY{n+nn}{numpy} \PY{k}{as} \PY{n+nn}{np}\PY{o}{,} \PY{n+nn}{math}
        \PY{k+kn}{import} \PY{n+nn}{matplotlib}\PY{n+nn}{.}\PY{n+nn}{pyplot} \PY{k}{as} \PY{n+nn}{plott}
\end{Verbatim}


    \begin{Verbatim}[commandchars=\\\{\}]
{\color{incolor}In [{\color{incolor}2}]:} \PY{k}{def} \PY{n+nf}{bucketSort}\PY{p}{(}\PY{n}{arr}\PY{p}{)}\PY{p}{:}
            \PY{n}{largest} \PY{o}{=} \PY{n+nb}{max}\PY{p}{(}\PY{n}{arr}\PY{p}{)}
            \PY{n}{length} \PY{o}{=} \PY{n+nb}{len}\PY{p}{(}\PY{n}{arr}\PY{p}{)}
            \PY{n}{size} \PY{o}{=} \PY{n}{largest}\PY{o}{/}\PY{n}{length}
         
            \PY{n}{buckets} \PY{o}{=} \PY{p}{[}\PY{p}{[}\PY{p}{]} \PY{k}{for} \PY{n}{\PYZus{}} \PY{o+ow}{in} \PY{n+nb}{range}\PY{p}{(}\PY{n}{length}\PY{p}{)}\PY{p}{]}
            \PY{k}{for} \PY{n}{i} \PY{o+ow}{in} \PY{n+nb}{range}\PY{p}{(}\PY{n}{length}\PY{p}{)}\PY{p}{:}
                \PY{n}{j} \PY{o}{=} \PY{n+nb}{int}\PY{p}{(}\PY{n}{arr}\PY{p}{[}\PY{n}{i}\PY{p}{]}\PY{o}{/}\PY{n}{size}\PY{p}{)}
                \PY{k}{if} \PY{n}{j} \PY{o}{!=} \PY{n}{length}\PY{p}{:}
                    \PY{n}{buckets}\PY{p}{[}\PY{n}{j}\PY{p}{]}\PY{o}{.}\PY{n}{append}\PY{p}{(}\PY{n}{arr}\PY{p}{[}\PY{n}{i}\PY{p}{]}\PY{p}{)}
                \PY{k}{else}\PY{p}{:}
                    \PY{n}{buckets}\PY{p}{[}\PY{n}{length} \PY{o}{\PYZhy{}} \PY{l+m+mi}{1}\PY{p}{]}\PY{o}{.}\PY{n}{append}\PY{p}{(}\PY{n}{arr}\PY{p}{[}\PY{n}{i}\PY{p}{]}\PY{p}{)}
         
            \PY{k}{for} \PY{n}{i} \PY{o+ow}{in} \PY{n+nb}{range}\PY{p}{(}\PY{n}{length}\PY{p}{)}\PY{p}{:}
                \PY{n}{insertion\PYZus{}sort}\PY{p}{(}\PY{n}{buckets}\PY{p}{[}\PY{n}{i}\PY{p}{]}\PY{p}{)}
         
            \PY{n}{result} \PY{o}{=} \PY{p}{[}\PY{p}{]}
            \PY{k}{for} \PY{n}{i} \PY{o+ow}{in} \PY{n+nb}{range}\PY{p}{(}\PY{n}{length}\PY{p}{)}\PY{p}{:}
                \PY{n}{result} \PY{o}{=} \PY{n}{result} \PY{o}{+} \PY{n}{buckets}\PY{p}{[}\PY{n}{i}\PY{p}{]}
        \PY{c+c1}{\PYZsh{}     print(result)}
            \PY{k}{return} \PY{n}{result}
         
        \PY{k}{def} \PY{n+nf}{insertion\PYZus{}sort}\PY{p}{(}\PY{n}{arr}\PY{p}{)}\PY{p}{:}
            \PY{k}{for} \PY{n}{i} \PY{o+ow}{in} \PY{n+nb}{range}\PY{p}{(}\PY{l+m+mi}{1}\PY{p}{,} \PY{n+nb}{len}\PY{p}{(}\PY{n}{arr}\PY{p}{)}\PY{p}{)}\PY{p}{:}
                \PY{n}{temp} \PY{o}{=} \PY{n}{arr}\PY{p}{[}\PY{n}{i}\PY{p}{]}
                \PY{n}{j} \PY{o}{=} \PY{n}{i} \PY{o}{\PYZhy{}} \PY{l+m+mi}{1}
                \PY{k}{while} \PY{p}{(}\PY{n}{j} \PY{o}{\PYZgt{}}\PY{o}{=} \PY{l+m+mi}{0} \PY{o+ow}{and} \PY{n}{temp} \PY{o}{\PYZlt{}} \PY{n}{arr}\PY{p}{[}\PY{n}{j}\PY{p}{]}\PY{p}{)}\PY{p}{:}
                    \PY{n}{arr}\PY{p}{[}\PY{n}{j} \PY{o}{+} \PY{l+m+mi}{1}\PY{p}{]} \PY{o}{=} \PY{n}{arr}\PY{p}{[}\PY{n}{j}\PY{p}{]}
                    \PY{n}{j} \PY{o}{=} \PY{n}{j} \PY{o}{\PYZhy{}} \PY{l+m+mi}{1}
                \PY{n}{arr}\PY{p}{[}\PY{n}{j} \PY{o}{+} \PY{l+m+mi}{1}\PY{p}{]} \PY{o}{=} \PY{n}{temp}
\end{Verbatim}


    \begin{Verbatim}[commandchars=\\\{\}]
{\color{incolor}In [{\color{incolor}3}]:} \PY{k}{def} \PY{n+nf}{measure\PYZus{}time}\PY{p}{(}\PY{n}{arr}\PY{p}{)}\PY{p}{:}
            \PY{n}{startTime} \PY{o}{=} \PY{n}{time}\PY{o}{.}\PY{n}{time}\PY{p}{(}\PY{p}{)}
            \PY{n}{bucketSort}\PY{p}{(}\PY{n}{arr}\PY{o}{.}\PY{n}{copy}\PY{p}{(}\PY{p}{)}\PY{p}{)}
            
            \PY{n}{duration} \PY{o}{=} \PY{n}{time}\PY{o}{.}\PY{n}{time}\PY{p}{(}\PY{p}{)} \PY{o}{\PYZhy{}} \PY{n}{startTime}
            \PY{k}{return} \PY{n}{duration}
        \PY{k}{def} \PY{n+nf}{measure\PYZus{}time\PYZus{}tim}\PY{p}{(}\PY{n}{arr}\PY{p}{)}\PY{p}{:}
            \PY{n}{arr1} \PY{o}{=} \PY{n}{arr}\PY{o}{.}\PY{n}{copy}\PY{p}{(}\PY{p}{)}
            \PY{n}{startTime} \PY{o}{=} \PY{n}{time}\PY{o}{.}\PY{n}{time}\PY{p}{(}\PY{p}{)}
            \PY{n}{arr1}\PY{o}{.}\PY{n}{sort}\PY{p}{(}\PY{p}{)}
            \PY{n}{duration} \PY{o}{=} \PY{n}{time}\PY{o}{.}\PY{n}{time}\PY{p}{(}\PY{p}{)} \PY{o}{\PYZhy{}} \PY{n}{startTime}
        \PY{c+c1}{\PYZsh{}     print(arr1)}
            \PY{k}{return} \PY{n}{duration}
        \PY{k}{def} \PY{n+nf}{increase}\PY{p}{(}\PY{p}{)}\PY{p}{:}
            \PY{n}{measures\PYZus{}arr\PYZus{}int64} \PY{o}{=} \PY{p}{[}\PY{p}{]}
            \PY{n}{measures\PYZus{}arr\PYZus{}int32} \PY{o}{=} \PY{p}{[}\PY{p}{]}
            \PY{n}{measures\PYZus{}arr\PYZus{}int8} \PY{o}{=} \PY{p}{[}\PY{p}{]}
            \PY{n}{measures\PYZus{}arr\PYZus{}int64\PYZus{}tim} \PY{o}{=} \PY{p}{[}\PY{p}{]}
            \PY{n}{measures\PYZus{}arr\PYZus{}int32\PYZus{}tim} \PY{o}{=} \PY{p}{[}\PY{p}{]}
            \PY{n}{measures\PYZus{}arr\PYZus{}int8\PYZus{}tim} \PY{o}{=} \PY{p}{[}\PY{p}{]}
            \PY{n}{f\PYZus{}x} \PY{o}{=} \PY{p}{[}\PY{p}{]}
            \PY{k}{for} \PY{n}{i} \PY{o+ow}{in} \PY{n+nb}{range}\PY{p}{(}\PY{l+m+mi}{1}\PY{p}{,}\PY{l+m+mi}{100}\PY{p}{)}\PY{p}{:}   
                \PY{n}{arr\PYZus{}int64} \PY{o}{=} \PY{n}{np}\PY{o}{.}\PY{n}{array}\PY{p}{(}\PY{n}{np}\PY{o}{.}\PY{n}{random}\PY{o}{.}\PY{n}{randint}\PY{p}{(}\PY{l+m+mi}{0}\PY{p}{,}\PY{l+m+mi}{100}\PY{p}{,}\PY{n}{i}\PY{p}{,} \PY{n}{dtype}\PY{o}{=}\PY{n}{np}\PY{o}{.}\PY{n}{int64}\PY{p}{)}\PY{p}{)}
                \PY{n}{arr\PYZus{}int32} \PY{o}{=} \PY{n}{arr\PYZus{}int64}\PY{o}{.}\PY{n}{astype}\PY{p}{(}\PY{n}{np}\PY{o}{.}\PY{n}{int32}\PY{p}{)}\PY{o}{.}\PY{n}{tolist}\PY{p}{(}\PY{p}{)}
                \PY{n}{arr\PYZus{}int8} \PY{o}{=} \PY{n}{arr\PYZus{}int64}\PY{o}{.}\PY{n}{astype}\PY{p}{(}\PY{n}{np}\PY{o}{.}\PY{n}{int8}\PY{p}{)}\PY{o}{.}\PY{n}{tolist}\PY{p}{(}\PY{p}{)}
                \PY{n}{arr\PYZus{}int64} \PY{o}{=} \PY{n}{arr\PYZus{}int64}\PY{o}{.}\PY{n}{tolist}\PY{p}{(}\PY{p}{)}
                \PY{c+c1}{\PYZsh{}int64 }
                \PY{n}{duration} \PY{o}{=} \PY{n}{measure\PYZus{}time}\PY{p}{(}\PY{n}{arr\PYZus{}int64}\PY{p}{)}
                \PY{n}{measures\PYZus{}arr\PYZus{}int64}\PY{o}{.}\PY{n}{append}\PY{p}{(}\PY{n}{duration}\PY{p}{)}
                
                \PY{n}{duration} \PY{o}{=} \PY{n}{measure\PYZus{}time\PYZus{}tim}\PY{p}{(}\PY{n}{arr\PYZus{}int64}\PY{p}{)}
                \PY{n}{measures\PYZus{}arr\PYZus{}int64\PYZus{}tim}\PY{o}{.}\PY{n}{append}\PY{p}{(}\PY{n}{duration}\PY{p}{)}
                \PY{c+c1}{\PYZsh{}int32 }
                \PY{n}{duration} \PY{o}{=} \PY{n}{measure\PYZus{}time}\PY{p}{(}\PY{n}{arr\PYZus{}int32}\PY{p}{)}
                \PY{n}{measures\PYZus{}arr\PYZus{}int32}\PY{o}{.}\PY{n}{append}\PY{p}{(}\PY{n}{duration}\PY{p}{)}
                
                \PY{n}{duration} \PY{o}{=} \PY{n}{measure\PYZus{}time\PYZus{}tim}\PY{p}{(}\PY{n}{arr\PYZus{}int32}\PY{p}{)}
                \PY{n}{measures\PYZus{}arr\PYZus{}int32\PYZus{}tim}\PY{o}{.}\PY{n}{append}\PY{p}{(}\PY{n}{duration}\PY{p}{)}
                \PY{c+c1}{\PYZsh{}int8}
                \PY{n}{duration} \PY{o}{=} \PY{n}{measure\PYZus{}time}\PY{p}{(}\PY{n}{arr\PYZus{}int8}\PY{p}{)}
                \PY{n}{measures\PYZus{}arr\PYZus{}int8}\PY{o}{.}\PY{n}{append}\PY{p}{(}\PY{n}{duration}\PY{p}{)}
                
                \PY{n}{duration} \PY{o}{=} \PY{n}{measure\PYZus{}time\PYZus{}tim}\PY{p}{(}\PY{n}{arr\PYZus{}int8}\PY{p}{)}
                \PY{n}{measures\PYZus{}arr\PYZus{}int8\PYZus{}tim}\PY{o}{.}\PY{n}{append}\PY{p}{(}\PY{n}{duration}\PY{p}{)}
        
            
            
            \PY{n}{plott}\PY{o}{.}\PY{n}{title}\PY{p}{(}\PY{l+s+s2}{\PYZdq{}}\PY{l+s+s2}{Analysis of time variation between int64,int32,int8 and built\PYZhy{}in sort algorithm}\PY{l+s+s2}{\PYZdq{}}\PY{p}{)}
            \PY{n}{plott}\PY{o}{.}\PY{n}{ylabel}\PY{p}{(}\PY{l+s+s1}{\PYZsq{}}\PY{l+s+s1}{Seconds}\PY{l+s+s1}{\PYZsq{}}\PY{p}{)}
            \PY{n}{plott}\PY{o}{.}\PY{n}{xlabel}\PY{p}{(}\PY{l+s+s1}{\PYZsq{}}\PY{l+s+s1}{Number of integers}\PY{l+s+s1}{\PYZsq{}}\PY{p}{)}
            
            \PY{n}{plott}\PY{o}{.}\PY{n}{plot}\PY{p}{(}\PY{n}{measures\PYZus{}arr\PYZus{}int64}\PY{p}{,} \PY{n}{label}\PY{o}{=}\PY{l+s+s1}{\PYZsq{}}\PY{l+s+s1}{int64}\PY{l+s+s1}{\PYZsq{}}\PY{p}{)}
            \PY{n}{plott}\PY{o}{.}\PY{n}{plot}\PY{p}{(}\PY{n}{measures\PYZus{}arr\PYZus{}int32}\PY{p}{,} \PY{n}{label}\PY{o}{=}\PY{l+s+s1}{\PYZsq{}}\PY{l+s+s1}{int32}\PY{l+s+s1}{\PYZsq{}}\PY{p}{)}
            \PY{n}{plott}\PY{o}{.}\PY{n}{plot}\PY{p}{(}\PY{n}{measures\PYZus{}arr\PYZus{}int8}\PY{p}{,} \PY{n}{label}\PY{o}{=}\PY{l+s+s1}{\PYZsq{}}\PY{l+s+s1}{int8}\PY{l+s+s1}{\PYZsq{}}\PY{p}{)}
            \PY{n}{plott}\PY{o}{.}\PY{n}{plot}\PY{p}{(}\PY{n}{measures\PYZus{}arr\PYZus{}int64\PYZus{}tim}\PY{p}{,} \PY{n}{label}\PY{o}{=}\PY{l+s+s1}{\PYZsq{}}\PY{l+s+s1}{int64\PYZus{}tim}\PY{l+s+s1}{\PYZsq{}}\PY{p}{)}
            \PY{n}{plott}\PY{o}{.}\PY{n}{plot}\PY{p}{(}\PY{n}{measures\PYZus{}arr\PYZus{}int32\PYZus{}tim}\PY{p}{,} \PY{n}{label}\PY{o}{=}\PY{l+s+s1}{\PYZsq{}}\PY{l+s+s1}{int32\PYZus{}tim}\PY{l+s+s1}{\PYZsq{}}\PY{p}{)}
            \PY{n}{plott}\PY{o}{.}\PY{n}{plot}\PY{p}{(}\PY{n}{measures\PYZus{}arr\PYZus{}int8\PYZus{}tim}\PY{p}{,} \PY{n}{label}\PY{o}{=}\PY{l+s+s1}{\PYZsq{}}\PY{l+s+s1}{int8\PYZus{}tim}\PY{l+s+s1}{\PYZsq{}}\PY{p}{)}
            
            \PY{n}{plott}\PY{o}{.}\PY{n}{legend}\PY{p}{(}\PY{p}{)}
            \PY{n}{plott}\PY{o}{.}\PY{n}{show}\PY{p}{(}\PY{p}{)}
        
        \PY{n}{increase}\PY{p}{(}\PY{p}{)}
\end{Verbatim}


    \begin{center}
    \adjustimage{max size={0.9\linewidth}{0.9\paperheight}}{output_3_0.png}
    \end{center}
    { \hspace*{\fill} \\}
    
    We can observe from the graph that with increasing the number of
integers in array, bucket sort starts to sort slower than the built-in
sorting algorithm - Timsort and this tendency can be observed with all
types of integers(int64,int32,int8). Generally it is slower 200-300
times with big-sized arrays. I came to this conclusion by dividing
duration of bucket sort algorithm in each occassion to 200 and 300 and
somewhere between this numbers the graphs are similar. On the other hand
among the int types the most slow one is int64. The second one is int32
and the fastest sorting int type is int8. After this experiment we can
make a conclusion that integers with smallest number of bytes are faster
than integers with big amount of bytes.

    EX2: To visualize the skip-list insertion I chose the simple software
visualization tool - Paint. In the first step I created the first level
of skip list with the smallest element of minus infinity(-∞) and then
put the first number(1) into the list.

    

    As the structure of skip list is randomized in this task we were obliged
to use coin to simulate the randomization. I chose the simple 20 cent
coin for it, where heads meant to move to upper level, and tails - to
stop. So first flipping ended up with heads, I moved to the upper level,
actually created it and flipped the coin for the second time and got the
heads, so I stopped.

    

    The actual process of insertion when it is processed in machine or
computer goes from upper level and searches by going everytime to the
bottom level, for example if we have to insert 5, we should search its
place from the top level by comparing it with values on each level,for
example, when 5 is less than the next value on the current level we move
to the bottom level, compare 5 to the values here and find its place
among the values. As I described the actual process of insertion,in the
next steps I will just insert the number to the bottom level without
giving the description of a whole process.

    

    I've inserted 5 and coin showed the tails two time and the flipped to
the heads, so now we have 3 levels with value 5 in each of them.

    

    On the next step I inserted 7 to our list and by flipping the coin I've
got heads from the very first try, so I stopped on the bottom level. How
many steps it took for us to insert 7? Firstly on the 3rd level we
compared our value with 5, and 7 was bigger than it. As 5 was the
biggest number in our list we come to the level1 and insert it right
after 5. Totally we have 4 steps with insertion itself.

    \begin{figure}
\centering
\includegraphics{attachment:Capture4.PNG}
\caption{Capture4.PNG}
\end{figure}

    The fourth value had the same actions as 5, so now we have 5 and 30 on
the third level.

    \begin{figure}
\centering
\includegraphics{attachment:Capture5.PNG}
\caption{Capture5.PNG}
\end{figure}

    As I described the process of insertion we are searching the place of
new element from the top level and for value 2 we found its place right
after 1, then coin showed the tails 3 times and then heads and we
stopped by creating the level 4 as I showed in the picture.

    \begin{figure}
\centering
\includegraphics{attachment:Capture6.PNG}
\caption{Capture6.PNG}
\end{figure}

    \begin{figure}
\centering
\includegraphics{attachment:Capture7.PNG}
\caption{Capture7.PNG}
\end{figure}

    \begin{figure}
\centering
\includegraphics{attachment:Capture8.PNG}
\caption{Capture8.PNG}
\end{figure}

    After insertion of 4,18,32,34 the the coin showed heads and now we have
these values only on the bottom level.

    \begin{figure}
\centering
\includegraphics{attachment:Capture9.PNG}
\caption{Capture9.PNG}
\end{figure}

    Insertion of 36 and coin reacts was the same as with 2, so now on the
4th level we have minus infinity,2 and 36.

    To find a vale we start as with insertion from top level. 35 is more
than 2, but less than 36, so we go down to the 3rd level. Here we
compare it with 5 and 30 and in both occasions 35 is more than these
values, but again less than 36, so we are moving to the second level.
Here again the next value is 36 and we repeat our moving down and now we
are on the bottom level. 35 is more than 32 and 34, but the next value
is again 36. As we are on the bottom level it means that we couldn't
manage to find our wished value. We performed 9 comparisons before
figuring out that we don't have 35 in our skip list.

    EX3: In this exercise we should insert the same values into binary
search tree in the same manner. To visualise it I used the same
software. The first 4 steps are skewed to the right. 1 is root.
5\textgreater{}1,7\textgreater{}5 and 30\textgreater{}7.

    \begin{figure}
\centering
\includegraphics{attachment:Capture_BST_step1.PNG}
\caption{Capture\_BST\_step1.PNG}
\end{figure}

    Then we should add '2' to the tree and now we start to compare. 2 is
bigger than 1, so we go to the right and less than 5, so we go to the
left.

    \begin{figure}
\centering
\includegraphics{attachment:Capture_BST_step2.PNG}
\caption{Capture\_BST\_step2.PNG}
\end{figure}

    Then we should insert 4 and 18. 4 is bigger than 1, less than 5 and
bigger than 2, so we build a new branch to the right of 2. 18 is bigger
than 1,5,7 but less than 30, so we build a new branch for 18 to the left
of 30.

    \begin{figure}
\centering
\includegraphics{attachment:Capture_BST_step3.PNG}
\caption{Capture\_BST\_step3.PNG}
\end{figure}

    The last numbers are all bigger than what we had in a tree and among
each other the next one is bigger than previous one, so we end up with
tree like that.

    \begin{figure}
\centering
\includegraphics{attachment:Capture_BST.PNG}
\caption{Capture\_BST.PNG}
\end{figure}

    As we can observe from the picture of out final tree, it is not complete
BST and it is skewed to the right, but also cannot be considered as the
right skewed binary tree because some of the nodes has 2 children.

    EX4: The general definition of AVL tree: it is a self-balancing Binary
Search Tree where the difference between heights of left and right
subtrees cannot be more than one for all nodes. So it means that when
the height of 2 children differs more than 1, then the tree should be
balanced. So, firstly when we insert the elements to the tree we have
this structure.

    \begin{figure}
\centering
\includegraphics{attachment:Capture_AVL_step1.PNG}
\caption{Capture\_AVL\_step1.PNG}
\end{figure}

    It contradicts to the defintion of AVL tree, because on the right we
already have 2 child nodes and none on the leftm so we re-balance the
tree.

    \begin{figure}
\centering
\includegraphics{attachment:Capture_AVL_step2.PNG}
\caption{Capture\_AVL\_step2.PNG}
\end{figure}

    Then we add 3 more numbers and our tree again becomes disbalanced. We
change the position of 2 and 1.

    \begin{figure}
\centering
\includegraphics{attachment:Capture_AVL_step3.PNG}
\caption{Capture\_AVL\_step3.PNG}
\end{figure}

    \begin{figure}
\centering
\includegraphics{attachment:Capture_AVL_step4.PNG}
\caption{Capture\_AVL\_step4.PNG}
\end{figure}

    \begin{figure}
\centering
\includegraphics{attachment:Capture_AVL_step5.PNG}
\caption{Capture\_AVL\_step5.PNG}
\end{figure}

    \begin{figure}
\centering
\includegraphics{attachment:Capture_AVL_step6.PNG}
\caption{Capture\_AVL\_step6.PNG}
\end{figure}

    Now our tree disbalanced again because height of 2 and 7 differs with
more than 1 branch(2). So we change the position of 7 and move it to the
top. Actually the left side after this movement also became disbalanced,
so 4 moved to the top and left branch is become balanced.

    \begin{figure}
\centering
\includegraphics{attachment:Capture_AVL_step7.PNG}
\caption{Capture\_AVL\_step7.PNG}
\end{figure}

    \begin{figure}
\centering
\includegraphics{attachment:Capture_AVL_step8.PNG}
\caption{Capture\_AVL\_step8.PNG}
\end{figure}

    After inserting 36 we should again balance our tree, as 32 has two more
children than 18. So we move 32 to 30th place and below it is the BST
with AVL-balancing.

    \begin{figure}
\centering
\includegraphics{attachment:Capture_AVL_step9.PNG}
\caption{Capture\_AVL\_step9.PNG}
\end{figure}

    To search 35 in this tree we perform 4 comparisons and figure out that
actually we don't have 35 in our tree(35\textgreater{}7,32,34 but less
than 36).

    EX5: Red black BST has the several properties or rules that should be
obeyed: 1.Tree is a valid binary search tree 2.All nodes are either red
or black 3.The root is black 4.All leaves are black 5.Every red node has
two black children 6.Every path from root to leaf contain the same
number of black nodes

    \begin{figure}
\centering
\includegraphics{attachment:red_black1.PNG}
\caption{red\_black1.PNG}
\end{figure}

    This is what we have after first insertion and as all requirements are
fulfilled we can move to the second step.

    \begin{figure}
\centering
\includegraphics{attachment:red_black2.PNG}
\caption{red\_black2.PNG}
\end{figure}

    We have inserted 5 and it is red because "Every path from root to leaf
contain the same number of black nodes".

    \begin{figure}
\centering
\includegraphics{attachment:red_black3.PNG}
\caption{red\_black3.PNG}
\end{figure}

    By inserting 7 now we have a conflict and we should rotate the tree,
move 5 to the top and change its color with 1. This is right right case.

    \begin{figure}
\centering
\includegraphics{attachment:red_black4.PNG}
\caption{red\_black4.PNG}
\end{figure}

    After insertion of 30 we have new kind of conflict. Every red node
should have two black children, but 7 does not. So we change the color
of 7 and 1. 1. 30's uncle is RED 2. Change color of 1 and 7 as BLACK. 3.
Change color of grand parent as RED. 4. Change 30 = 30's grandparent.

    \begin{figure}
\centering
\includegraphics{attachment:red_black5.PNG}
\caption{red\_black5.PNG}
\end{figure}

    \begin{figure}
\centering
\includegraphics{attachment:red_black6.PNG}
\caption{red\_black6.PNG}
\end{figure}

    Inserting 2 happenned without a conflict.

    \begin{figure}
\centering
\includegraphics{attachment:red_black7.PNG}
\caption{red\_black7.PNG}
\end{figure}

    Now we insert 4.

    \begin{figure}
\centering
\includegraphics{attachment:red_black8.PNG}
\caption{red\_black8.PNG}
\end{figure}

    and again we have "Every red node has two black children" conflict", so
we move 2 to the upper level, change its and 1's color. Left Right Case.

    \begin{figure}
\centering
\includegraphics{attachment:red_black9.PNG}
\caption{red\_black9.PNG}
\end{figure}

    \begin{figure}
\centering
\includegraphics{attachment:red_black10.PNG}
\caption{red\_black10.PNG}
\end{figure}

    By inserting 18 we have the red node which has only one black node, so
again we should rotate. Right right case.

    \begin{figure}
\centering
\includegraphics{attachment:red_black11.PNG}
\caption{red\_black11.PNG}
\end{figure}

    now 18 is red and on the second level.

    \begin{figure}
\centering
\includegraphics{attachment:red_black12.PNG}
\caption{red\_black12.PNG}
\end{figure}

    With inserting 32 we don't have any trouble.

    \begin{figure}
\centering
\includegraphics{attachment:red_black14.PNG}
\caption{red\_black14.PNG}
\end{figure}

    With inserting 34, 32 has only one black child node. Right right case.

    \begin{figure}
\centering
\includegraphics{attachment:red_black13.PNG}
\caption{red\_black13.PNG}
\end{figure}

    We rotate 32 and 30 and change theirs color.

    The final step was inserting 36 and it totally messed up the structure
that I was trying to apply.

    \begin{figure}
\centering
\includegraphics{attachment:red_black15.PNG}
\caption{red\_black15.PNG}
\end{figure}

    Here 34 had only one black node, but as the structure of tree becomes
skewed to the rightm I decided to change the top node and make it 18.

    \begin{figure}
\centering
\includegraphics{attachment:red_black16.PNG}
\caption{red\_black16.PNG}
\end{figure}

    Changed 18's color to black and remain only 1,4 and 36 red. This tree
obey to all 6 rules that are prescripted to black-red trees. I rotated
the tree after 6 insertions.


    % Add a bibliography block to the postdoc
    
    
    
    \end{document}
